\documentclass[10pt, a4paper]{article}

\usepackage[utf8]{inputenc}
\usepackage[italian]{babel}
\usepackage[
	a4paper,
	top=2.5cm,
	bottom=2cm,
	left=2cm,
	right=2cm
]{geometry}
\usepackage{graphicx}
\usepackage{fancyhdr}
\usepackage[
	colorlinks=true,
	linkcolor=black,
	urlcolor=blue
]{hyperref}
\usepackage{makecell}
\usepackage{float}
\usepackage[
	format=plain,
	labelfont=it,
	textfont=it
]{caption}
\usepackage{ifxetex}
\ifxetex
	\usepackage{fontspec}
	\setmainfont{Arial}
\else
	\usepackage{helvet}
	\renewcommand{\familydefault}{\sfdefault}
\fi
\usepackage{amsmath}

%\usepackage[inline]{showlabels} % debugging purposes

\graphicspath{ {./pic} }

\linespread{1} %interlinea

\def\arraystretch{1.5}
\renewcommand\theadfont{\bfseries}

\newcommand{\Intmaketable}[4]{
	\begin{table}[h!]
	\centering
	\begin{tabular}{#3}
	#4
	\end{tabular}
	\caption{#2}
	\label{#1}
	\end{table}
}

\newcommand{\Intceqtable}[3]{
	\Intmaketable{#1}{#2}{|l|l|}{
	\hline
	\thead{Parametro formale} & \thead{Classi d'equivalenza} \\
	\hline
	\hline
	#3
	\hline}
}

\newcommand{\Intbvtable}[3]{
	\Intmaketable{#1}{#2}{|l|l|l|}{
	\hline
	\thead{Parametro formale} & \thead{Classe d'equivalenza} & \thead{Boundary value}\\
	\hline
	\hline
	#3
	\hline}
}

\newcommand{\Inttestctable}[3]{
	\Intmaketable{#1}{#2}{|l|l|l|}{
	\hline
	\thead{Input} & \thead{Esito atteso} & \thead{Motivazione}\\
	\hline
	\hline
	#3
	\hline}
}

\newcommand{\Intceqcaption}[4]{Classi d'eq. per metodo #1 di #2, iter. #3 (#4)}
\newcommand{\Intbvcaption}[4]{Boundary values per classi d'eq. metodo #1 di #2, iter #3 (#4)}
\newcommand{\Inttestccaption}[4]{Casi di test per metodo #1 di #2, iter. #3 (#4)}

%% BEGIN COMMANDS TO BE USED

\newcommand{\gettablelabel}[5]{table:#1:#2:#3:iter#4:proj#5}

\newcommand{\ceqtable}[5]{
	\Intceqtable{\gettablelabel{ceq}{#1}{#2}{#3}{#4}}
		{\Intceqcaption{#1}{#2}{#3}{#4}}
		{#5}
}

\newcommand{\bvtable}[5]{
	\Intbvtable{\gettablelabel{bv}{#1}{#2}{#3}{#4}}
		{\Intbvcaption{#1}{#2}{#3}{#4}}
		{#5}
}

\newcommand{\testctable}[5]{
	\Inttestctable{\gettablelabel{testc}{#1}{#2}{#3}{#4}}
		{\Inttestccaption{#1}{#2}{#3}{#4}}
		{#5}
}

\newcommand{\getpicturelabel}[1]{picture:#1}

\newcommand{\makepicture}[4]{
	\begin{figure}[h!]
	\centering
	\includegraphics[width=#1, height=#2]{#3}
	\caption{#4}
	\label{\getpicturelabel{#3}}
	\end{figure}
}

\newcommand{\ctrdfcovcaption}[4]{Contatori copertura dataflow (def-use) per metodo #1 di #2, iter. #3 (#4)}
\newcommand{\alldfcovcaption}[4]{Vista completa per data-flow coverage (def-use) per metodo #1 di #2, iter. #3 (#4)}
\newcommand{\methodcfcovcaption}[4]{Statement coverage e branch coverage per metodo #1 di #2, iter. #3 (#4)}
\newcommand{\classcfcovcaption}[3]{Statement coverage e branch coverage per classe #1, iter. #2 (#3)}
\newcommand{\mutclasscaption}[3]{Statistiche sulla mutazione dell'intera classe #1, iter. #2 (#3)}
\newcommand{\mutmethodcaption}[4]{Mutanti killed (in verde più scuro), survived (in rosso più scuro) per il metodo #1 di #2, iter. #3 (#4)}
\newcommand{\finishcodecaption}[4]{#2.#1, iter. #3 (#4)}

\newcommand{\tcell}{\makecell[tl]}
\newcommand{\newtrow}{\\ \hline}

%% END OF COMMANDS TO BE USED

\def\reporttitle{Report ISW2 modulo testing}
\def\authorsname{Belli Stefano}
\def\univcode{0350116}

\title{\reporttitle}
\author{\authorsname\;(\univcode)}
\date{}

\pagestyle{fancy}
\fancyhf{}
\renewcommand{\headrulewidth}{0pt}

\fancyhead[LO,LE]{\authorsname}
\fancyhead[RO,LE]{\univcode}
\fancyfoot[CO,RE]{\thepage}

\begin{document}

	\maketitle
	\thispagestyle{empty}

	\tableofcontents

	\section*{Obiettivo del documento}
	L’obiettivo del documento è quello di illustrare tutte le attività svolte e le decisioni prese al fine ottenere una test suite
	adeguata e robusta, relativamente ai due progetti BookKeeper e Storm (Apache Software Foundation). 
	Per entrambi i progetti è stato impostato il framework di continuous integration GitHub Actions – con un file YAML, 
	al cui interno viene specificato il SO, le “actions” che ad esempio si occupano di effettuare il checkout (git clone) del repo, 
	installare una certa versione di java, e i comandi da eseguire nella VM/container istanziato, causato da un trigger (es. push di
	commit al repo remote). 
	Tramite l’ausilio di maven (in particolare i profili) e script scritti ad-hoc per il continuous integration, è stato possibile
	testare calcolando anche le metriche di adeguatezza control-flow, data-flow e effettuare mutazioni sui test. 
	I risultati vengono collezionati e caricati come artefatti della build 
	(file html/xml di JaCoCo, PIT e ba-dua compressi in un archivio zip).

	\pagebreak

	\section{Progetto BookKeeper}
	Il readme del repository \href{https://github.com/apache/bookkeeper}{upstream} ne fornisce 
	la seguente definzione: \textit{“Apache BookKeeper is a scalable, 
	fault-tolerant and low latency storage service optimized for append-only workloads”}. 
	La documentazione di \href{https://bookkeeper.apache.org/archives/docs/master/bookkeeperOverview.html}{overview}
	fornisce una visione a grandi linee del sistema: 
	è necessario acquisirne almeno una minima conoscenza per individuare correttamente le classi da testare. 
	Descrivere il sistema non è coerente con lo scopo di questo documento, qualora fosse necessario ai fini 
	di motivare le scelte e le decisioni intraprese per individuare classi d’equivalenza, boundary analysis,\dots\,
	si riprenderanno i concetti di BookKeeper.
	Il source tree è diviso in vari artifact maven (ognuno ha un proprio pom.xml, la root ha il proprio pom.xml), 
	si decide di testare le classi dell'artifact bookkeeper-server. \\
	
	\subsection{org.apache.bookkeeper.net.NetworkTopologyImpl}
	Si decide di testare la classe perchè essendo BookKeeper un sistema di ledger \underline{distribuiti}, è di fondamentale importanza l'organizzazione
	corretta dei "bookies" (server BookKeeper) secondo una certa topologia di rete. La documentazione dell'
	\href{https://bookkeeper.apache.org/docs/latest/api/javadoc/org/apache/bookkeeper/net/NetworkTopology.html}{interfaccia NetworkTopology} specifica i
	metodi che NetworkTopologyImpl deve necessariamente implementare. La classe implementor di nostro interesse (NetworkTopologyImpl) è documentata 
	\href{https://bookkeeper.apache.org/docs/latest/api/javadoc/org/apache/bookkeeper/net/NetworkTopologyImpl.html}{qui} e si legge:
	\textit{"The class represents a cluster of computer with a tree hierarchical network topology. 
	For example, a cluster may be consists of many data centers filled with racks of computers. In a network topology, 
	leaves represent data nodes (computers) and inner nodes represent switches/routers that manage traffic in/out of data centers or racks."}.
	Per capire meglio, cercando nel web, ci si imbatte nella documentazione di 
	\href{https://bookkeeper.apache.org/docs/latest/api/javadoc/org/apache/bookkeeper/client/EnsemblePlacementPolicy.html}{EnsemblePlacementPolicy} 
	che contiene una spiegazione più dettagliata della "Network Topology": 
	\textit{"The network topology is presenting a cluster of bookies in a tree hierarchical structure. 
	For example, a bookie cluster may be consists of many data centers (aka regions) filled with racks of machines. 
	In this tree structure, leaves represent bookies and inner nodes represent switches/routes that manage traffic in/out of regions or racks.
	For example, there are 3 bookies in region `A`. They are `bk1`, `bk2` and `bk3`. And their network locations are /region-a/rack-1/bk1, 
	/region-a/rack-1/bk2 and /region-a/rack-2/bk3."} e rende tutto più chiaro. Infatti, tornando all'
	\href{https://bookkeeper.apache.org/docs/latest/api/javadoc/org/apache/bookkeeper/net/NetworkTopology.html}{interfaccia NetworkTopology}, vediamo che i metodi
	manipolano la rete con i \href{https://bookkeeper.apache.org/docs/latest/api/javadoc/org/apache/bookkeeper/net/Node.html}{Node} e dalla documentazione di
	quest'ultima interfaccia viene riportato come: \textit{"The interface defines a node in a network topology. A node may be a leave representing a data node 
	or an inner node representing a datacenter or rack. Each data has a name and its location in the network is decided by a string with syntax similar to a file name. 
	For example, a data node's name is hostname:port\# and if it's located at rack "orange" in datacenter "dog", the string representation of its network location is /dog/orange"}.
	Quindi, concettualmente si ha una rete strutturata ad albero: 
	\begin{itemize}
		\item c'è il nodo root
		\item i nodi interni (inner nodes) che rappresentano i router "associabili" a regions, datacenters, racks, \dots
		\item le foglie dell'albero (leaves) che rappresentano i bookies veri e propri
	\end{itemize}

	Ogni nodo ha un nome e si può raggiungere specificandone il path: il carattere di separazione '/' indica i nodi di cui fare traversal prima
	di raggiungere il nodo finale (il percorso parte dalla root e passa per i nodi specificati dalla stringa path, il percorso deve esistere nell'albero o viene creato, dettagli a seguire)

	Se considerassimo ad esempio il ramo 
	$$ \text{ROOT} \rightarrow \text{datacenter}_1 \rightarrow \text{rack}_1 \rightarrow \text{bookie}_1$$

	\begin{itemize}	
		\item /: indica la ROOT
		\item /datacenter-1: indica l'inner node che rappresenta il router del datacenter-1
		\item /datacenter-1/rack-1: indica l'inner node che rappresenta il router del rack-1 (via datacenter-1)
		\item /datacenter-1/rack-1/bookie-1: indica il bookie (foglia, via datacenter-1 e rack-1)
	\end{itemize}
	
	Qualsiasi altro path, in questo contesto è invalido (nel senso che per questo semplice albero, costituito da questo singolo ramo,
	le uniche combinazioni per raggiungere qualsiasi nodo dell'albero sono queste 4).
	Un nodo foglia non può avere come parent il nodo ROOT, all'atto pratico questo significa che il bookie server deve essere necessariamente
	collocato in un rack/datacenter/\dots (collegato quindi a un router che rappresenti questo rack/datacenter/\dots), inserire un nodo tipo
	\textit{"/bookie-1"} (essendo bookie$_1$ il server bookie/nodo foglia) non è consentito. Inserire invece \textit{"/rack-1/bookie-1"} va bene
	, cosi come \textit{"/datacenter-1/rack-1/bookie-1"} e cosi via, incrementando l'altezza dell'albero\dots

	\subsubsection{Prima iterazione}
	Si decide di testare i metodi 
	\href{https://bookkeeper.apache.org/docs/latest/api/javadoc/org/apache/bookkeeper/net/NetworkTopologyImpl.html#add(org.apache.bookkeeper.net.Node)}{add} e 
	\href{https://bookkeeper.apache.org/docs/latest/api/javadoc/org/apache/bookkeeper/net/NetworkTopologyImpl.html#remove(org.apache.bookkeeper.net.Node)}{remove}
	di \href{https://bookkeeper.apache.org/docs/latest/api/javadoc/org/apache/bookkeeper/net/NetworkTopologyImpl.html}{NetworkTopologyImpl} 
	perchè sono ritenuti i più importanti in quanto permettono di manipolare la topologia di rete aggiungendo e togliendo nodi.
	L'implementazione dell'interfaccia
	\href{https://bookkeeper.apache.org/docs/latest/api/javadoc/org/apache/bookkeeper/net/Node.html}{Node} scelta è
	\href{https://bookkeeper.apache.org/docs/latest/api/javadoc/org/apache/bookkeeper/net/BookieNode.html}{BookieNode} che oltretutto eredita da 
	\href{https://bookkeeper.apache.org/docs/latest/api/javadoc/org/apache/bookkeeper/net/NodeBase.html}{NodeBase} 
	che comunque implementa, ovviamente, Node. E un'altra sua implementazione è
	\href{https://bookkeeper.apache.org/docs/latest/api/javadoc/org/apache/bookkeeper/net/NetworkTopologyImpl.InnerNode.html}{NetworkTopologyImpl.InnerNode} (che eredita da NodeBase).
	\\
	Dato che i test di unità vanno effettuati senza considerare l'implementazione e "mettendosi dalla parte dell'utilizzatore", viene scelta BookieNode perchè è l'implementazione di 
	Node utilizzata dalle classi client che interagiscono con NetworkTopology (al di fuori del package org.apache.bookkeeper.net), mentre non vengono usate le altre due perchè appunto: 
	NodeBase serve a implementare altri tipi di "Node" e InnerNode è una classe statica all'interno di NetworkTopologyImpl, oltretutto con visibilità package-private.
	Comunque scegliere implementazioni piuttosto che altre non è di fondamentale importanza in quanto le NetworkTopology nei loro metodi accettano come parametri formali l'interfaccia Node
	(ai quali tutte le implementazioni devono aderire).

	\subsection{org.apache.bookkeeper.bookie.FileInfo}

\end{document}